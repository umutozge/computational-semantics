\documentclass[10pt,a4paper]{exam}

\usepackage{umut}
\usepackage{uling}
\usepackage{mathptmx}
\usepackage{uprog}
\usepackage{uhref}
\usepackage{usynsem}

%\printanswers

\pagestyle{headandfoot}
	\lhead{Cogs 543 \\ Computational Semantics}
	\chead{Assignment 4}
	\rhead{Fall 2023\\ due Jan 2, before class}
\lfoot{}
\pointname{\%}

\begin{document}
\qformat{\bf Q \thequestion.%
\ifthenelse{\equal{\thepoints}{}}{}{\quad (\thepoints)} \hfill}

% \makebox[\textwidth]{Name of the Student:\enspace\hrulefill}

\vspace{10pt}

% \begin{center}
% \fbox{\parbox{6in}{\bf\centering 4 questions in 150 minutes}}
% \end{center}
%

\vspace{90pt}


Here is your lexicon:


\begin{align*}
	\text{walks}   &:=& \sysm{\lambda x.walk'x}                          &::& \sysm{et}\\[8pt]
	\text{loves}   &:=& \sysm{\lambda x\lambda y.love'xy}                &::& \sysm{e(et)}\\
	\text{reads}   &:=& \sysm{\lambda x\lambda y.read'xy}                &::& \sysm{e(et)}\\[8pt]
	\text{John}    &:=& \sysm{\lambda p.p\cnct{}j'}                      &::& \sysm{ett}\\
	\text{Mary}    &:=& \sysm{\lambda p.p\cnct{}m'}                      &::& \sysm{ett}\\[8pt]
	\text{woman}   &:=& \sysm{\lambda x.woman'x}                         &::& \sysm{et}\\
	\text{book}    &:=& \sysm{\lambda x.book'x}                          &::& \sysm{et}\\[8pt]
	\text{blue}    &:=& \sysm{\lambda p\lambda x.blue'x\land p\cnct{}x}  &::& \sysm{et(et)}\\[8pt]
	\text{is}      &:=& \sysm{\lambda p\lambda x.p (\lambda x.x=x) x}    &::& \sysm{et(et)(et)}\\[8pt]
	\text{no}      &:=& \sysm{\lambda p\lambda q.\neg(\exists x. px \land qx)} &::& \sysm{et(ett)}\\
	\text{a}       &:=& \sysm{\lambda p\lambda q.\exists x. px \land qx} &::& \sysm{et(ett)}\\
	\text{every}   &:=& \sysm{\lambda p\lambda q.\forall x. px \cond qx} &::& \sysm{et(ett)}\\[8pt]
	\text{ACC}     &:=& \sysm{\lambda k \lambda q \lambda y.k (\lambda x.q\cnct{} x\cnct{} y)} &::& \sysm{et(ett)(e(et)et)}\\
	\text{NOM}     &:=& \sysm{\lambda p.p} &::& \sysm{ett(ett)}\\
\end{align*}

\begin{questions}

\question[20] The lexicon above is one of the possible models of copula
and adjectival semantics. It also models two case markers the
\emph{nominative} and the \emph{accusative}. Make sure that the
lexicon can deliver what it aims to deliver. Indicate any errors you
find. Do not think taht there \emph{must} be errors. The lexicon may
just be fine. Those like the following structures should be interpretable
correctly.


\ex.
\a. ((NOM John) (reads (ACC (every (blue book))))).
\b. ((NOM Every woman) (reads (ACC (every (blue book))))).


\question

You are required to propose a new lexicon, so that an expression like:


\ex.\label{knife} John killed Mary with a knife.

receives the meaning,


\ex.\label{knife-interp} \sysm{\lambda e. killing'e\land agent'e\cnct{j'}\land patient'e\cnct{s'}\land \exists x. knife'x \land instr'e\cnct{}x}


Notice that (\ref{knife-interp}) is not a type $t$ interpretation, it
is rather $vt$, where $v$ is the type of eventualities. Therefore,
(\ref{knife-interp}) denotes a set of event(ualitie)s.


You need to model two alternative approaches:

\begin{parts}
	\part[40] the predicates \sysm{agent'} and \sysm{patient'} are
	contributed by the verb \emph{kill};
	\part[40] they come from case marking.
\end{parts}

in either case the predicate \sysm{instr'} will be contributed by
\emph{with}.

\end{questions}
\end{document}
