\documentclass[10pt,a4paper]{exam}

\usepackage{umut}
\usepackage{uling}
\usepackage{usynsem}
\usepackage{mathptmx}
\usepackage{uprog}
\usepackage{uhref}

\printanswers

\pagestyle{headandfoot}
	\lhead{Cogs 543 \\ Computational Semantics}
	\chead{Assignment 03}
\rhead{Fall 2020 \\ due Jan 15, Friday}
\lfoot{}
\pointname{\%}

\begin{document}
\qformat{\bf Question \thequestion%
\ifthenelse{\equal{\thepoints}{}}{}{\quad (\thepoints)} \hfill}

% \makebox[\textwidth]{Name of the Student:\enspace\hrulefill}

\vspace{10pt}

% \begin{center}
% \fbox{\parbox{6in}{\bf\centering 4 questions in 150 minutes}}
% \end{center}
% 

\vspace{90pt}

\begin{questions}

\question Write the needed lexical categories and derive the logical forms of the following sentences:

\begin{parts}
\part Every woman who John loves likes cats and walks. (Take \emph{cats} as an \sysm{NP:cats'} of type \sysm{e}, a plural individual.)  
\part Some woman who John loves and Mary likes cats.
\end{parts}

\end{questions}
\end{document}

\question The task is to generate a random model on the basis of a set of entities $E$ and a vocabulary, which is a mapping from names to arities. Arity is the number of empty slots in a function. Zero arity means a term, like a proper name; an arity of 1 indicates a property; an arity of 2 indicates a 2 place relation, and so on.\footnote{The concept of arity applies to predicates and functions themselves; here, however, we are using it as it is a property of the names of predicates and functions. This is merely for convenience, things will get clear in due course.}  

You will need to bring together a domain of entities and a vocabulary and randomly generate a model: a structure that associates 0-arity names with individuals in the domain, 1-arity names with subsets of the domain, and 2-arity names with set of ordered pairs from the domain, and so on, all randomly. 

If you haven't solved the last week's assignment totally, (i) try to complete it; but if you can't do so, you may (ii) use the solution for that assignment (please see \Verb+code/util/setutils.lisp+) in your solution for this week's assignment.  In case you do (ii), please study the solutions carefully.

Here is a sample output of what is expected for the assignment. Of course, your solution may look different than this, depending on your language or strategy.

\begin{ucodeframe}
\begin{Verbatim}
The domain of entities: (G385 G386 G387 G388 G389 G390 G391)

0-place names: 

JOHN:   G389
MARY:   G391
BILL:   G391
SUE:    G386

1-place names: 

HAPPY:  (G386 G388 G389 G385)
GREEN:  (G390 G388 G389 G386 G387 G385)
BOOK:   NIL
DOG:    (G391 G387 G388 G385)
CAT:    (G391 G389 G390 G385 G386 G387 G388)
HUMAN:  (G390 G388)
SLEEPS: (G387 G385 G390 G391 G388 G386 G389)
WALKS:  (G385)

2-place names: 

LOVES:  ((G390 G386) (G387 G391) (G391 G388) (G389 G387) (G389 G390) (G390 G387)
        (G387 G385) (G389 G385) (G386 G387) (G391 G391) (G391 G387) (G387 G386)
        (G388 G391) (G390 G389))
HATES:  ((G390 G389) (G391 G391))
CHASES: ((G387 G389) (G390 G389) (G385 G385) (G391 G391) (G388 G390)
         (G388 G391) (G389 G387) (G387 G386) (G388 G388) (G385 G390)
         (G388 G389))
\end{Verbatim}
\end{ucodeframe}


